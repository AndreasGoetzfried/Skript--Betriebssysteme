\documentclass{article}
\usepackage{graphicx}
\usepackage[utf8]{inputenc}
\usepackage{listings}
\usepackage{color}
\usepackage{amsfonts}
\usepackage{tabularx}
\usepackage{mathtools}
\usepackage[ngerman]{babel}
\newcommand\norm[1]{\left\lVert#1\right\rVert}

\begin{document}
\begin{titlepage}
\centering
    \begin{figure}
    \centering
	    \includegraphics[width=90mm]{logo_lmu.jpg}
    \end{figure}
	{\scshape\LARGE Ludwig-Maximilians Universität \par}
	\vspace{1cm}
	{\scshape\Large Skript \par}
	\vspace{1.5cm}
	{\huge\bfseries Betriebssysteme\par}
	\vspace{2cm}
	{\Large\itshape Andreas Götzfried\par}
    \vfill
	    basierend auf\par
	    Prof. Dr. C.\textsc{Linnhoff-Popien}
    \vfill
	{\large \today\par}
\end{titlepage}
\tableofcontents{}

\newpage
\section{Einführung}
\subsection{Das Betriebssystem}
\subsubsection{Einordnung der Maschinensprache}
\subsubsection{Aufgaben des Betriebssystems}
\subsubsection{Geschichte der Betriebssysteme}
\subsubsection{Arten von Betriebssystemen}

\newpage
\section{Prozesse}
\subsection{Programme und Unterprogramme}
\subsubsection{Vom Programm zum Maschinenprogramm}
\subsubsection{Unterprogramme und Prozeduren}
\subsubsection{Realisierung eines Unterprogrammaufrufs}
\subsubsection{Rekursive Prozeduraufrufe}
\subsection{Prozesse}
\subsubsection{Das Prozess-Konzept}
\subsubsection{Prozessbeschreibung}
\subsubsection{Prozesskontrolle}
\subsection{Threads}
\subsubsection{Multithreading}
\subsubsection{Threadzustände}
\subsubsection{User-Level-Threads (ULT)}
\subsubsection{Kernel-Level-Threads (KLT)}
\subsubsection{Kombinierte Konzepte}
\subsubsection{Andere Formen paralleler Abläufe}
\subsection{Scheduling}
\subsubsection{Das Prinzip des Schedulings}
\subsubsection{Scheduling-Algorithmen}
\subsubsection{Prozesswechsel}
\subsubsection{Arten des Schedulings}

\newpage
\section{Multiprocessing}
\subsection{Deadlocks bei Prozessen}
\subsubsection{Motivation der Deadlocks anhand zweier Beispiele}
\subsubsection{Das Prinzip der Deadlocks}
\subsubsection{Deadlock Prevention}
\subsection{Prozesskoordination}
\subsubsection{Nebenläufigkeit von Prozessen}
\subsubsection{Kritische Bereiche}
\subsubsection{Wechselseitiger Ausschluss}
\subsubsection{Semaphore}
\subsubsection{Monitore}
\subsubsection{Message Passing}

\newpage
\section{Ressourcenverwaltung}
\subsection{Speicher}
\subsubsection{Speicherverwaltung}
\subsubsection{Speicherpartitionierung}
\subsubsection{Virtueller Speicher}
\subsubsection{Paging}
\subsubsection{Segmentierungsstrategien}
\subsection{E/A Verwaltung}
\subsubsection{Klassifizierung von E/A-Geräten}
\subsubsection{E/A Techniken}

\newpage
\section{Interprozesskommunikation}
\subsection{Lokale Interprozesskommunikation}
\subsubsection{Grundlagen des Nachrichtenaustauschs}
\subsubsection{Pipes}
\subsubsection{FIFOs}
\subsubsection{Stream Pipes}
\subsubsection{Sockets}
\subsection{Verteilte Systeme}
\subsubsection{Einführung in Verteilte Systeme}
\subsubsection{Kommunikation in Verteilte Systeme}
\end{document}